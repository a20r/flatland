
\title{Probabilistic Audio Localization in Wireless Acoustic Sensor Networks}

\date{\today}

\author{Alexander Wallar}

\documentclass[12pt]{article}

\usepackage[pdftex]{graphicx}

\usepackage{amsmath}

\usepackage{relsize}

\usepackage{float}

\usepackage{algorithm}

\usepackage[noend]{algorithmic}

\floatstyle{ruled} \newfloat{program}{thp}{lop} \floatname{program}{Structure}

\newcommand{\Normal}[3]{\mathcal{N}(#1, #2, #3)}

\newcommand{\Acronym}[1]{\ensuremath{{\small{\texttt{#1}}}}}
\newcommand{\Name}{\Acronym{Camgaze.js}} \newcommand{\False}{\Constant{false}}
\newcommand{\True}{\Constant{true}}
\newcommand{\Symbol}[1]{\ensuremath{\mathcal{#1}}}
\newcommand{\Function}[1]{\ensuremath{{\small \textsc{#1}}}}
\newcommand{\Constant}[1]{\ensuremath{\small{\texttt{#1}}}}
\newcommand{\Var}[1]{\ensuremath{{\small{\textsl{#1}}}}}
\newcommand{\argmin}[1]{\underset{#1}{\operatorname{arg}\,\operatorname{min}}\;}
\newcommand{\grad}[1]{\underset{#1}{\operatorname{\Function{GradientDecent}}}\;}

\begin{document}

\maketitle

Dimensionality reduction is a key component of making robotic motion planning
fast and efficient. A rigid body has six degrees of freedom (DoF), so a
multi-link robot such as a humanoid or snake robot can have dozens or hundreds
of DoF. The rotational and translational transformations of a rigid body can be
described with the 3-dimensional Special Euclidean group (known as SE(3)).
SE(3) is homeomorphic to the topological space R3xRP3 (where RP3 is the
3-dimensional real projective plane). Therefore it is easy to imagine a
multi-link robot with an arbitrarily complex state space. The picture is
further complicated by the addition of obstacles into a robot's world. The
state space for motion planning, which must take into account both the robot
and obstacles, is called the configuration space. 

Motion planning essentially consists of a search over the configuration space
from a start configuration to a goal configuration. A huge number of methods
for searching through the configuration space have been developed, most of
which can be divided into two classes, sampling-based motion planning and
combinatorial motion planning. In sampling-based motion planning, obstacles in
the C-space are defined implicitly so that the path is constructed by randomly
or pseudorandomly sampling points from the C-space and doing collision
checking. In combinatorial motion planning, obstacles are defined explicitly
and a complete search of the C-space is made. Because of the difficulty of
explicitly defining all of the geometry in a potentially complex world,
sampling-based motion planning techniques such as the Rapidly-exploring Random
Tree (RRT) and Probabilistic RoadMap (PRM) planning are the most popular motion
planning techniques.

[RRT* algorithm]
[PRM* algorithm]

We used RRT* and PRM* in our project. The * indicates that these are the
asymptotically optimal versions of RRT and PRM (they are guaranteed to find the
optimal path as time goes to infinity). [Description of how RRT* and PRM* work]

Despite the efficiency of sampling-based search, the high dimensionality of the
configuration space of complex robots means that motion planning for many
robots is slow or infeasible. This is because of the "curse of dimensionality",
in which the size of the search space increases exponentially with the
dimension. One way to speed up motion planning is to reduce the dimensionality
of the C-space and do search in this reduced dimensionality space. For this
project, we tested the performance of a number of dimensionality reduction
techniques to see which one was the most effective in speeding up motion
planning while providing reliable obstacle avoidance.

We used FeatureAgglomeration, TruncatedSVD, PCA, KernelPCA, RandomizedPCA, and
our own "Trained Johnson-Lindestrauss" to perform dimensionality reduction.

PCA: Principle Component Analysis, or PCA, transforms a set of variables in
their "principle components", or linearly uncorrelated variables.

ncollisions, duration, length, num failed. Run each one 100 times

\cite{wallar2015reactive}

\begin{algorithm}[ht] 
    \caption{Multiple Source Sound Localization}
    \label{algo:MS}
    \begin{algorithmic}[1]
        \setcounter{ALC@line}{0}
        \vspace*{1mm}

        \STATE $\Function{GetLocations}(\Var{r}, \Var{spl}, \Var{eventList}) \rightarrow$
        \STATE $\Var{locations} \leftarrow \Function{InitLocationList}()$
        \FORALL {$e \in \Var{eventList}$}
        \STATE $\Var{locations}.\Var{add}(\mathcal{L}_m(\Var{r}, \Var{spl}, \Var{eventList}, e.\Var{x}, e.\Var{y}))$
        \ENDFOR
        \STATE $\Var{centers} \leftarrow \Function{AffinityPropagation}(\Var{locations})$
        \COMMENT {\Var{centers} is a list of $(x, y, c)$ points where $c$ is the confidence}
        \RETURN $\Var{centers}$

    \end{algorithmic}
\end{algorithm}

\bibliographystyle{plain}
\bibliography{bib}

\end{document}
